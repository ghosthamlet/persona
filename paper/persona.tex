\def\year{2020}\relax
%File: formatting-instruction.tex
\documentclass[letterpaper]{article} %DO NOT CHANGE THIS
\usepackage{times}  %Required
\usepackage{helvet}  %Required
\usepackage{courier}  %Required
\usepackage{url}  %Required
\usepackage{graphicx}  %Required
\usepackage{natbib}
\usepackage{hyperref}       % hyperlinks
% https://mirrors.tuna.tsinghua.edu.cn/CTAN/macros/latex/contrib/hyperref/doc/backref.pdf
\usepackage[hyperpageref]{backref}

\usepackage{amsmath}
\usepackage{IEEEtrantools}

% \usepackage[sort&compress,square,comma,numbers]{natbib}

\usepackage{CJKutf8}
\AtBeginDvi{\input{zhwinfonts}}

% all custom newcommand should begin with prefix: K
\newcommand{\KModelName}{RAR模型}
\newcommand{\KFactorNoused}{\underline{时间、对话环境、对方地点、发言者地点}}

% \DeclareRobustCommand{\citeext}[1]{\citeauthor{#1}~\cite{#1}}
\DeclareRobustCommand{\citeext}[1]{\cite[#1]{#1}}

\frenchspacing  %Required
\setlength{\pdfpagewidth}{8.5in}  %Required
\setlength{\pdfpageheight}{11in}  %Required
\setcounter{secnumdepth}{0}  
\linespread{1.2}

\begin{document}
\begin{CJK*}{UTF8}{gbsn}

\title{开放领域个性化对话生成专用模型的改进}
\author{Jing Bo Hu, Qiang Han \& Hua Wei Liu \\
HuggingFace  Inc.\\
{\tt \{hu,gvvvv,liu\}@163.com} \\}

\maketitle
\begin{abstract}
本论文主要分析Transformer架构近几年来的一些速度和性能上的改进方法,其在专用模型训练上的运用,这里研究的专用模型是指个性感知的开放领域对话生成模型,针对的数据集是多轮短对话, 单个输入系列总长度不超过90个字,因此Transformer架构在长文本处理上架构、注意力机制等方面的诸多改进不在本论文讨论之列。
\end{abstract}

\section[Background]{背景} 
NLP(自然语言处理)领域的革命从注意力机制\citeext{Bahdanau2015}的奠基,Transformer架构\citeext{Vaswani2017}点燃了导火索,到BERT模型\citeext{Devlin2019}正式拉开革命帷幕,以简单的纯注意力机制架构取代了曾经是NLP主流的相对复杂的RNN架构系列模型,并在模型性能、扩展性以及速度上都取得了极大的提升,4年间全面改变了NLP领域,造就了NLP领域的ImageNet时刻\citeext{ruder2018nlpimagenet}。

开放领域对话生成是NLP的一项重要且十分复杂的任务,Transformer架构在这项任务上同样取得了质的飞跃,如GPT-2和GPT-3模型\citeext{Radford2019, Brown2020}。后文如无指定说明,涉及模型皆为Transformer架构模型。

GPT-2、GPT-3模型都是通用模型,适用于几乎所有NLP任务,并非专用于开放领域对话生成,这类通用模型需以巨大的模型尺寸、海量预训练数据和漫长的预训练时间为代价,才能在性能上与专用中小模型相匹配(尽管有DistilGPT2\footnote{\url{https://github.com/huggingface/transformers/tree/master/examples/distillation}},DistilBERT\citeext{Sanh2019}对通用模型的优化,但其性能也相应下降)。
而且通用模型很难生成个性一致的对话\footnote{即聊天机器人带有明确固定个性特征如姓名性别地址等等的个性化对话},即使如微调GPT-2模型\citeext{Radford2019}的方法TransferTransfo\citeext{Wolf2019}\footnote{在输入的对话上下文中增加了个性化数据},虽然其在个性化对话英文数据集PERSONA-CHAT\citeext{Zhang2018}上表现不错,但在个性化对话中文数据集PersonalDialog\citeext{Zheng2019a}上测试时仍然不如专用模型\citeext{Zheng2019}。

本论文即对开放领域个性化对话生成专用模型\citeext{Zheng2019}引入Transformer架构近几年的速度和性能上的一些改进,并分析在此专用模型上有效和无效的各种尝试,实验的源代码已经开源\footnote{\url{https://github.com/ghosthamlet/persona}}。后文简称开放领域个性化对话生成为个性化对话生成。

\section[Related Works]{相关研究} 
个性化对话生成的研究相对于无个性对话生成和闭合领域任务型对话生成来说比较少,因此相关的对话数据集也较少。英文的主要是PERSONA-CHAT\citeext{Zhang2018},中文的有PersonalDialog\citeext{Zheng2019a}和Personality Assignment\citeext{Qian2017}。这里讨论的多数研究主要是使用这三个数据集。

早期的对话生成基本是RNN+注意力机制\citeext{Bahdanau2015}的Seq2Seq模型\citeext{SutskeverGoogle2014},如复杂的多阶段训练的\citeext{Qian2017}(数据集Personality Assignment),简单的端到端训练的\citeext{Zheng2019a}(数据集PersonalDialog)等等,也有RNN+内存机制\citeext{Sukhbaatar2015}的Seq2Seq模型,如\citeext{Zhang2018}(数据集PERSONA-CHAT)中的几个基线模型。

随着Transformer架构的兴起,近来大部分模型都已经是Transformer架构原生的通用的或经过更改组合的专用模型,如更改GPT\citeext{Radford2018}的模型\citeext{Tselousov2018}(数据集PERSONA-CHAT),原生GPT的\citeext{Wolf2019}(数据集PERSONA-CHAT),加入Attention Routing的\citeext{Zheng2019}(数据集PersonalDialog),组合GPT和BERT的\citeext{Liu2020}(数据集PERSONA-CHAT)等等。

此外还有相对少量的利用VAE(Variational AutoEncoder),RL(Reinforcement Learning)和GAN(Generative Adversarial Network)等等的模型,如CVAE(Conditional VAE)+RNN+内存机制的\citeext{Song2019}(数据集PERSONA-CHAT),VAE+GRU(RNN)+内存机制+注意力机制的\citeext{Xu2020}(数据集PERSONA-CHAT),RL+Transformer的\citeext{Liu2020}(数据集PERSONA-CHAT)等等,GAN相关的模型我们未作研究,在此不做介绍。

我们选用Transformer架构的Attention Routing模型\citeext{Zheng2019}作为研究对象,是基于其模型的简单性及高效,并且未偏离原生Transformer架构太远,因而能够对其运用大多数原生Transformer架构的改进。另外我们使用的数据集是PersonalDialog,但由于资源有限暂无法利用其完整的数百万数据,训练与评测只在随机分别取出的10万会话上进行。全数据集的训练与评测留待将来研究。后文简称Attention Routing模型为AR模型,我们改进的模型称为AR+模型。

\section[Model]{模型} 
AR模型为Encoder Decoder结构,类型于Transformer架构的完整版\citeext{Vaswani2017},不同点是:


AR+模型在保留AR模型整体结构基础上,引入了以下改进:


\section[Experiments]{实验} 
\subsection[Datasets]{数据集} 

\section[Ablation Study]{消融研究} 

\section[Conclusion]{结论} 
揭示预训练在大规模标记数据和专用模型上的适用问题。

\section[Acknowledgement]{致谢}

\bibliographystyle{plain}
\bibliography{./persona.bib}
\clearpage\end{CJK*}
\end{document}

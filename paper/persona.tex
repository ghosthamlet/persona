\def\year{2020}\relax
%File: formatting-instruction.tex
\documentclass[letterpaper]{article} %DO NOT CHANGE THIS
\usepackage{times}  %Required
\usepackage{helvet}  %Required
\usepackage{courier}  %Required
\usepackage{url}  %Required
\usepackage{graphicx}  %Required
\usepackage{natbib}

\usepackage{amsmath}
\usepackage{IEEEtrantools}

%% install LaTex: http://www.tug.org/texlive/quickinstall.html
%% For chinese: https://tex.stackexchange.com/questions/17611/how-does-one-type-chinese-in-latex
% \usepackage{CJKutf8}
% \AtBeginDvi{\input{zhwinfonts}}
% \begin{document}
% \begin{CJK*}{UTF8}{gbsn}
%   chinese texts
% \clearpage\end{CJK*}
% \end{document}

\usepackage{CJKutf8}
\AtBeginDvi{\input{zhwinfonts}}

% all custom newcommand should begin with prefix: K
\newcommand{\KModelName}{UCC模型}
\newcommand{\KFactorNoused}{\underline{时间、对话环境、对方地点、发言者地点}}

\frenchspacing  %Required
% \setlength{\pdfpagewidth}{8.5in}  %Required
% \setlength{\pdfpageheight}{11in}  %Required
\setcounter{secnumdepth}{0}  

\begin{document}
\begin{CJK*}{UTF8}{gbsn}

\title{更像人而不是更智能:对话系统的统一}
\author{Jing Bo Hu, Qiang Han \& Hua Wei Liu \\
HuggingFace  Inc.\\
{\tt \{hu,gvvvv,liu\}@163.com} \\}

\maketitle
\begin{abstract}
  \ldots
\end{abstract}

\section{简介} 
本论文主要分析Transformer模型近几年来的一些速度和性能上的改进方法,其在专用模型上的运用,这里研究的专用模型是指个性感知的开放领域对话模型,针对的数据集是多轮短对话,单个输入系列总长度不超过90个字,因此Transformer模型在长文本处理上的诸多改进不在本论文考虑之列。
